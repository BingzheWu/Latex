\part{一些准备}
\pagestyle{empty}
\section{概率论方面的一些概念}
\begin{define}
($\sigma$field)设$\Omega$是试验S的样本空间,用F表示$\Omega$
的某些子集构成的集合,如果$\mathscr{F}$满足:
\begin{enumerate}
\item $\Omega \in \mathscr{F}$;
\item 若$A\in \mathscr{F}$,则$\overline{A}\in \mathscr{F}$
\item 若$A_j\in \mathscr{F}$,则$\displaystyle\bigcup _{n=1}^{\infty}A_j\in \mathscr{F}$
\end{enumerate}
就称$\mathscr{F}$是$\Omega$的事件域或$\sigma$域,称$\mathscr{F}$中的元素为事件,
($\Omega,\mathscr{F}$)为可测空间
\end{define}
\begin{define}
(probability space)设($\Omega,\mathscr{F}$)为可测空间,P是定义在
$\mathscr{F}$上的函数,如果P满足下列条件:
\begin{enumerate}
\item 非负性:
对于$A\in \mathscr{F},P(A)\geq 0$
\item 完全性:
$P(\Omega)=1$;

\item 可列可加性:
对于$\mathscr{F}$中互不相容的事件$A_1,A_2,\cdots$,
\[
P( \bigcup^{\infty}_{j=1}A_j)=\sum_{j=1}^{\infty}P(A_j)
\]

\end{enumerate}
就称P为$\mathscr{F}$上的概率测度,简称概率,称$(\Omega,\mathscr{F},P)$为概率空间
\end{define}
\begin{define}
(conditionnal probability)Probability of an event A given that an event B has happened
 is denoted by $ P(A|B)$ and is computed
by
\[
P(A|B)=\dfrac{P(A\cap B)}{P(B)},P(B)>0
\]
Event A is said to be independent from event B if:
\[
P(A|B)=P(A)
\]
\end{define}
\begin{define}
(Random Variables)设$(\Omega,\mathscr{F})$是可测空间
,如果$\Omega$上的函数$X(\omega)$满足:对任何实数x,
\[
\{\omega|X(\omega)\leqslant x\}\in \mathscr{F}
\]
\end{define}
\begin{define}
(随机变量的独立性)设$X_1,X_2,\cdots$是随机变量.
\begin{enumerate}
\item
如果对于任何实数$x_1,x_2,\cdots,x_n$,
\begin{equation}
\begin{split}
& P(X_1\leqslant x_1,X_2\leqslant x_2,\cdots ,X_n\leqslant x_n)\\
& =P(X_1\leqslant x_1)P(X_2\leqslant x_2)\cdots P
(X_n\leqslant x_n)
\end{split}
\end{equation}
则称随机变量$X_1,X_2,\cdots,X_n$相互独立
\item 如果对任何n,$X_1,X_2,\cdots,X_n$相互独立,则称随机变量序列$\{X_i\}$为独立序列
\end{enumerate}
\end{define}

\begin{define}
(probability distribution function,PDF)对随机变量X,则称函数
\[
F(x)=P(X\leqslant x)
\]
为X的概率分布函数。有如下两条性质:
\begin{enumerate}
\item F单调不减右连续;
\item $F(\infty)=1,F(-\infty)=0$
\end{enumerate}
\end{define}
\begin{define}
(pdf,probability density function)设X为随机变量,如果存在非负函数$f(x)$使得对
任何a<b,
\[P(a<X\leqslant b)=\int_{a}^{b}f(x)dx\]
就称X是连续性随机变量,称$f(x)$为X的概率密度函数
\end{define}
\begin{define}

\begin{enumerate}


\item (mean,expection) $\langle X \rangle=E(x)=\int_{-\infty}^{-\infty}xf(x)dx$
\item (Variance) $Var(X)=E(X-\langle X\rangle)^2=EX^2-(EX)^2$
\item (k-th moment) $m_k(X)=E(X^k)$
\end{enumerate}
\end{define}
\begin{define}
\begin{enumerate}
\item (joint distribution of) Let X and Y be two random variables. The joint distribution of
X and Y is defined by:
\[F_{XY}(x,y)=P(X\leqslant x,Y\leqslant y)\]
\item (joint density) defined by:
\[f_{XY}(x,y)=\dfrac{\partial^2}{\partial x\partial y}F(x,y)\]
\item Notice that marginal distributions are:
\[F(x)=F_{XY}(x,\infty)\]
\item X and Y are independent identically distributed (i.i.d.) if
they are independent and have the same distribution.
\end{enumerate}
\end{define}
\begin{define}
\begin{enumerate}
\item (covariance)The covariance $C_{XY}$ of X and Y is defined be :
\[C_{XY}=E(X-E(X))E(Y-E(Y)) \]
\item The correlation coefficient of X and Y is defined be
\[\rho_{XY}=\dfrac{C_{XY}}{\sigma_x\sigma_y}\]
\item 
X and Y are uncorrelated if:
\[C_{XY}=0,or E(XY)=EXEY\]

\end{enumerate}
\end{define}
\begin{define}
随机变量函数函数的分布函数

设$\eta (\xi)$为随机变量$\xi$的函数,$F(x)$为$\xi$的分布函数,则$
\eta$的分布函数为$G(x)=\int_{\eta(y)<x}dF(y)$(由此定义可以直接得到阅读的文献中Chapter2中的Hilbert公式)
\end{define}

