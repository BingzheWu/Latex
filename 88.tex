\documentclass{article}
\usepackage{CJK}
\usepackage{amsmath}
\usepackage{amsfonts}
\usepackage{amssymb}
\usepackage{makeidx}
\usepackage{graphicx}
\usepackage{soul}
\begin{CJK}{UTF8}{gbsn}
\newtheorem{thm}{定理}
\newtheorem{zm}{证明}
\author{吴秉哲}
\title{整函数与亚纯函数}
\date{\today}

\begin{document}
\maketitle
若函数$f(z)$在$C$上解析,则称$f(z)$是一\emph{整函数}.对整函数$f(z)$
它的Taylor展开式
\begin{equation}
f(z)=\sum_{n=0}^{\infty}c_nz^n
\end{equation}
在$C$上成立.另一方面$z=\infty$为$f(z)$的孤立奇点,由Laurent展式.它的
奇异部分$\varphi(z)=f(z)$,解析部分$\varphi(z)=0$.

\begin{thm}
若整函数$f(z)$在$\overline{C}$上解析,则$f(z)$为常数.
\end{thm}

\begin{zm}
条件表明$f(z)$在为一整函数,且$\lim\limits_{z\rightarrow\infty}f(z)=c_0$存在
,所以$z=\infty$为$f(z)$在无穷远点的可去奇点,故$f(z)$的奇异部分$\varphi(z)=f(z)$
应为常数.证毕.
\end{zm}

定理也可以利用连续函数$\vert{f(z)}\vert$在紧集$\overline{C}$上
某一点$z_0$取到它的最大值,再根据最大模定理的$f(z)$为一常数.

若$z=\infty$是整函数的$m$级零点,则$f(z)$为一个$m$次的多项式.

若$z=\infty$为$f(z)$的本性奇点,则$f(z)$为\emph{超越整函数},
例如$e^z,\sin(z),\cos(z)$都是超越整函数.

若函数$f(z)$在区域$D\subset\overline{C}$上除去极点外解析,则
称$f(z)$是$D$内的\emph{亚纯函数}或\emph{半纯函数}.
由极点定义知极点是孤立的,所以$D$上的亚纯函数的极点集不可能有属于
$D$的极限点.例如函数$\tan(z)$是$C$上的亚纯函数,有理函数
\[
R(z)=P_n(z)\smallsetminus Q_m(z),
\]
其中$P_n(z),Q_m(z)$分别是$n,m$次多项式,称$max(n,m)$为有理函数
$R(z)$的次数,则有理函数是$C$上的亚纯函数,因为在$C$上至多有$m$个
极点.事实上它也是$\overline{C}$上的亚纯函数,

\end{CJK}

\end{document}