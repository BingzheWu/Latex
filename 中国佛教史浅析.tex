\documentclass[10pt,a4paper]{ctexart}
\usepackage[utf8]{inputenc}
\usepackage{amsmath}
\usepackage{amsfonts}
\usepackage{amssymb}
\usepackage{graphicx}
\author{吴秉哲}
\title{中国佛教建筑发展史简介及建筑背后的原因浅析}
\begin{document}
	\maketitle
\subsection*{内容简介及写作目的}
上大学三年以来,对我影响比较大的事情是分别在大二,大一两年暑假完成了五千公里的骑行,一年在西北,一年在西南。
基于这两次骑行,我体会了全新的旅行方式,见识了很多震撼的景象。在今年佛教史的课上,老师要求写论文,我脑子里冒出来的第一个想法就是路上去
看过的,路过的这些佛教的寺庙,以及其他一些与佛教有关的建筑(比如去年骑行到勐海县看到的八角亭),也自然地想起了这些建筑形态的一些差异,这些想法也成了我写这篇文章的初衷,我想通过去查阅一些资料结合中国佛教发展历史对这些建筑的形态有一个解释,通过一段时间的资料阅读,自己对中国佛教建筑
的发展历史以及它们背后的一些事情有了整体的认知,对他们的整合以及自己的一些想法便是下面的一些内容。

以上便是这篇文章的写作初衷以及目的,下面说一下文章的主要内容。主要分为两部分,其中一部分介绍中国佛教建筑的发展演变历史,其中自然包含了中国佛教的传入与发展历史,第二部分主要探究历史上的哪些因素(地理,文化,时代背景等)影响了佛教建筑的发展。最后一部分举了一些典型的实例。
\subsection*{中国佛教建筑的发展演变历史}
\subsubsection*{佛教的传入与发展}
\subsubsection*{中国佛教建筑历史的发展}
\subsection*{中国佛教}
\end{document}
