\documentclass[10pt,a4paper]{ctexart}
\usepackage[utf8]{inputenc}
\usepackage{longtable}
\usepackage{amsmath}
\usepackage{amsfonts}
\usepackage{amssymb}
\usepackage{graphicx}
\author{吴秉哲}
\title{太极理论作业}
\begin{document}
	\maketitle
	\section{体育锻炼与健康的关系}
	谈到这个问题,就我个人而言,上大学之后,有两件事情对我产生了很重要的影响,一次是大一暑假作为一名队员参与了
	北京大学自行车协会的暑期远征,路线是从内蒙古到嘉峪关,行程超过两千公里,期间还翻越了祁连山脉;还有一次是自己担任
	队长,带领着14名队员,完成了从张家界到西双版纳的远征,行程两千四百公里。这两次长途骑行,不仅大大强健了自己的体质,
	还改变了自己对生活的一些看法。具体来看,我觉得在远征途中,是我大学三年作息时间最规律,也是睡眠质量最好的一段时间。
	我觉得是由于每天都保证了足够的运动量,而且饮食均衡的缘故。通过我的测量,在长途骑行结束之后的一段时间,我的心率都会比自己
	之前的心率要慢,这也是身体强壮的一个体现吧。除了骑行之外,我平时也会参加一些徒步的活动,既可以看看美景,也可以锻炼身体。
	而且我明显感觉,有了适量的运动过后,睡眠的质量以及第二天人的精力都会有一定的提高,自己学习的效率,思考的效率也会得到
	改善。所以我认为选择一项或几项自己喜爱的运动,并持之以恒,会给我们的身体健康与心理健康都带来积极的影响。
	\section{太极拳24式名称}
	下面依照课堂教学的顺序列出24式名称:起势,野马分鬃,白鹤亮翅,搂膝拗步,手挥琵琶,倒卷肱,左揽雀尾,右揽雀尾
	,单鞭,云手,单鞭,高探马,右蹬脚,双峰贯耳,转身左蹬脚,左下势独立,右下势独立,左右穿梭,海底针,闪通臂,转身搬拦捶,
	如封似闭,十字手,收势
	\section{个人的锻炼计划}
	\begin{center}
		\begin{longtable}{|c|c|}
			\caption{个人锻炼计划}\\
			\hline
			时间& 活动\\
			\hline
			周一& 晚上长跑5圈\\
			\hline
			周二& 靠墙马步3组(每组2分钟)\\
			\hline
			周三& 晚上长跑5圈\\
			\hline
			周末& 长度超过100公里的骑行\\
			\hline
		\end{longtable}
	\end{center}
\end{document}