
\maketitle\documentclass[10pt,a4paper]{article}
\usepackage{CJK}
\usepackage[utf8]{inputenc}
\usepackage{amsmath}
\usepackage{amsfonts}
\usepackage{amssymb}
\usepackage{graphicx}
\newtheorem{thm}{定理}
\newtheorem{prove}{证明}
\begin{CJK}{UTF8}{gbsn}
\author{吴秉哲}
\title{实变函数证明合集}
\begin{document}
\section{集合与点集}
\begin{thm}
$[0,1]={x:0\leqslant x \leqslant 1}$不是可数集.
\end{thm}
\begin{prove}
只需讨论$(0,1]$,为此,采用二进制小数表示法:
\[
x=\sum^{\infty}_{n=1} \dfrac{a_n}{2^n}
\]
其中$a_n$等于0或1,并在表达式中存在无穷多个$a_n$等于1.显然
,$(0,1]$与全体二进制数一一对应.

若在上述表示中,把$a_n$=0的项舍去,则得到$x=\sum^{\infty}_{i=1}2^{-n_i},$这里的${n_i}$是严格上升的自然数数列.再令
$k_1=n_1,k_i=n_i-n_{i-1},i=2,3...,$
则$k_i$是自然数子列.把由自然数构成的序列的全体记为$\wp$,则$\wp$
与$(0,1]$一一对应.

现在假定$(0,1]$是可数的,则$\wp$是可数的,不妨将其全体排列如下:

\end{prove}
\end{CJK}
\end{document}