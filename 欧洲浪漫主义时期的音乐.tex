\documentclass[10pt,a4paper]{ctexart}
\usepackage[utf8]{inputenc}
\usepackage{amsmath}
\usepackage{amsfonts}
\usepackage{amssymb}
\usepackage{graphicx}
\usepackage{cite}
\usepackage{natbib}
\newcommand{\upcite}[1]{\textsuperscript{\textsuperscript{\cite{#1}}}}
\author{吴秉哲}
\title{《特里斯坦与伊索尔德》中的爱与悲}
\begin{document}
	\maketitle
	《特里斯坦与伊索尔德》是德国作曲家兼剧作家瓦格纳(1813-1883)成熟时期最有代表性的一部作品,是根据
	爱尔兰的传说撰写的爱情悲剧,瓦格纳创作的基本素材来源于13世纪德国诗人弗列特.冯.斯特拉斯布格的同名叙
	事诗\upcite{百度百科}。虽然其构思来源于古代传奇,但瓦格纳在这部作品中抛弃了以往的魔幻内容,魔鬼,神怪,圣杯骑士在这部作品
	中统统消失不见。瓦格纳将史诗中的情节也大大简化,变成了一部专门描写主人公恋爱内心世界的感情戏,台词大
	多为主人公的内心独白,音乐也大多表现纠缠于爱欲之中的痛苦和幻想。
	
	全局分为三幕,讲述了年轻骑士特里斯坦被派往爱尔兰,去为其叔父—康沃尔国王马克迎娶公主伊索尔德。特里斯坦与伊索尔德
	曾经相识,归涂中公主郁郁寡欢,因为她偷爱着特里斯坦,但他对她却冷淡无情,伊索尔德又想到自己被迫无奈的婚姻愤恨至极,
	宁肯与特里斯坦同归于尽。他命令其女仆在酒里下毒,但是女仆却将毒酒换为了爱情之酒,二人陷入情网。两人在一次深夜幽会中,
	被国王马克撞见,国王手下的骑士梅洛特向特里斯坦挑战,决斗中,特里斯坦身受重伤,他的仆人将其带往其家乡养伤。伊索尔德
	乘船赶到特里斯坦的家乡,特里斯坦非常激动,但还是死于伊索尔德的怀抱中。随后,国王乘船赶到本想成全特里斯坦与伊索尔德
	的爱情,但为时已晚。此时,伊索尔德已唱着爱情之歌,同特里斯坦共相长眠\upcite{麦耶尔}。
	
	整部歌剧是围绕着特里斯坦与伊索尔德两人爱情的产生,发展以及死亡来展开的,自由选择的死亡引领彼此相爱,同时也需要使
 	也需要爱情使两人走向死亡,这就注定了男女主人公命运的悲剧性。
 	
 	下面从几个方面来谈谈《特里斯坦和伊索尔德》创作背景中的悲剧创作元素以及歌剧本身蕴含的爱情与悲情。
 	\section*{创作背景}
 	\subsection*{音乐时代背景}
 	十九世纪中后期的音乐发展正处于浪漫派后期,浪漫主义音乐比起之前的维也纳古典乐派的音乐,更注重感情和形象的表现,相对
 	来说要看轻形式和结构一些。浪漫主义音乐的一个重要特点就是善于表达个人的感情和幻想,尤其强调个人主义的体验。他们热衷
 	于反映内心最强烈情感的爱情主题;倾心于带自传性的,不满现状的忧郁,孤独者的精神\upcite{余志刚}
 	
 	浪漫主义音乐的出现,使音乐走出宫廷,走向民间。新的艺术题材的出现更加丰富了音乐的表现力。这时的艺术家们不像在古典乐派
 	时期沐浴在阳光下,而是沉浸在充满不详预感,倾向于表现内心的思考之中,因而瓦格纳的音乐中充满了矛盾,不安,灵魂的超越
 	以及爱情的伟大等。
 	\subsection*{写作背景}
    瓦格纳在完成《特里斯坦与伊索尔德》时,已经是位在异国漂流近10年的政治流亡者。瓦格纳所处的年代,正是德国社会发生剧烈动荡的
    年代。当时瓦格纳曾热情投入1849年德国资产阶级革命,可是他站在了失败的一方,普鲁士国王的军队镇压了起义,瓦格纳受到通缉和流放
    ,他的那些带有乌托邦色彩的社会理想在社会现实面前被碰得粉碎,这是50年代后期(也就是他创作这部歌剧的时期)经历一场严重精神危机
    的根本原因。这场"革命"不仅没有改变他的命运,却使他更加烦恼,痛苦。他在1855年的《论事本》上写到:“我心里渴望死亡,而生活——是我痛恨的。”
    \upcite{瓦格纳书信集}。这便是50年代后期瓦格纳所面临的精神生活的危机。

    当时,瓦格纳在感情生活上也经历着一场严重的危机。这种危机是瓦格纳同一位有夫之妇—玛蒂尔德.韦森东克之间一段没有指望的恋情。瓦格纳
    在逃亡瑞士时,结识了富商欧托.韦森东克一家。其夫人玛蒂尔德是个作家,爱好音乐与文学。瓦格纳经常与她讨论文学艺术问题,当时的瓦格纳
    与他的妻子米娜感情不和,几乎闹到决裂。各种因素的作用下,瓦格纳与玛蒂尔德之间产生了恋情。对于瓦格纳自己来说,这是一场不可能有结果的爱情。
    一方面是自己心理过不去,另一方面是社会舆论的压力。当时与妻子不和的瓦格纳独自一人迁居威尼斯,并投入到《特里斯坦与伊索尔德》的创作之中。
    瓦格纳在给李斯特的信件中曾经说:“虽然我一生从未品尝过爱的甘美,我却要为这个所有梦中最美的梦树立一座纪念碑。在这个梦中,爱得到了永恒的满
    足。”\upcite{瓦格纳李斯特}。由此可以明显看出瓦格纳这段无结果的恋情给这部歌剧带来的影响,这些经历无疑给瓦格纳创作特里斯坦这一角色带来了
    灵感,伊索尔德便是他心目中的玛蒂尔德。也正是这些经历,注定了这部歌剧的悲情色彩,同样也体现了瓦格纳在现实中对爱情的追求。
 	\section*{第三幕终场音乐—“爱之死”分析}
    《爱之死》是歌剧《特里斯坦与伊索尔德》第三幕的终场音乐,伊索尔德在特里斯坦逝去时,也是在自己弥留之际的一段绝唱,它与歌剧中的另一个重要部分——《前奏曲》前后呼应,反应了瓦格纳当时对伦理,爱情,死亡等事物的看法。下面我分别从剧情和音乐上谈谈自己对终场音乐的看法。
    \subsection*{剧情—“永恒之爱”}
    从整部歌剧的剧情来看,最开始是伊索尔德的爱激发了特里斯坦的爱(爱的毒酒),但是她并没有使特里斯坦完全摆脱世俗的束缚,特里斯坦始终处在一种
    矛盾的状态,这种心态也与当时瓦格纳创作这部歌剧时的心态类似。甚至可以说,当第一幕特里斯坦喝下那杯“爱之酒”时,就已经注定了最后特里斯坦
    死亡的结局。特里斯坦这种矛盾的心理贯穿了整部歌剧,比如在第二幕,特里斯坦与梅洛特决斗时,要借梅洛特的剑来结束自己的生命的时候,体现了
    特里斯坦当时希望用死亡来化解他心中的矛盾。自从那杯酒后,特里斯坦便不能用逃避爱的方法来避开他自己心中的矛盾,这也注定了故事最后的结局,
    只能用死亡来结束这一切。另一方面,特里斯坦之死使伊索尔德的幻想破灭,于是伊索尔德全部的爱的热情便幻化到天堂,她选择了追随特里斯坦的脚步
    ,其实这个结局从前面的一些地方也可以看出,正如伊索尔德在第三幕中唱的那样:“这可怕的结局早在我的渴望中。”\upcite{生与悲的爱情}
    现在,她所渴望的东西发生了,于是伊索尔德咏诵出了《爱之死》。它与通常的歌剧在人物即将死亡的时候表现悲伤不同,它所表现的是两个主人公在
    灵魂与精神上的一种解脱,这是一个很有意思的地方,这里蕴含这一些宗教的思想(比如基督教中的“升入天堂”),值得一提的是,瓦格纳在整部歌剧中
    都体现了很多宗教的元素,比如第二幕的黑夜的描绘,体现了具有宗教色彩的神秘。整部戏中对黑夜,死亡,矛盾的描写都有强烈的悲观主义色彩。
    我想瓦格纳当时所要想表现的想法就是虽然特里斯坦与伊索尔德之间的爱情从世俗的眼光来看是违反道德的,但是他们之间是真正的爱情,是超出世俗
    意义的,最后只能通过死亡的方式,将真正的爱情留在永恒的地方。
    \subsection*{台词分析}
    “他在微笑,温柔动情,他的眼睛微微闪动”\upcite{西洋著名歌剧剧作集1}

    这是《爱之死》最开始的唱句,其旋律低而弱,渐起渐伏,然后稳定地增长到高潮,然后平息...
    前一部分已经说明了终曲不同于一般歌剧对死亡的描写(比如安魂曲),这里对死亡的描写更像是
    对生命升华的赞美。特里斯坦死后,将不再陷入世俗的矛盾。台词中有这么一段:

    朋友,看!你还没看见?

    也没听见远处的音乐

    多么柔和,多么动听?

    它在倾诉欢乐哀怨,

    它在表明和解宽容,

    沁我肺腑,动我心灵,

    环绕着我不断上升!\upcite{西洋著名歌剧剧作集1}
                
                    (《第三幕第三场》)
    
    从台词看出,瓦格纳想象中的天堂的样子,在天堂可以袒露心扉,不需要像在尘世的任何顾忌,再也没有任何的矛盾
    和冲突,第一句“你还没看见?”,表示了这个天堂的世界只有伊索尔德能够看见,那里的歌声也只有伊索尔德能够
    听见,因为特里斯坦去了那个世界,是他们之间的爱情赋予了她这些东西,而最后一句“环绕这我不断上升”表现了伊索尔德
    马上也会走进那个世界,这样随着情节的推动,渐渐引出了女主角的死亡:
    
    明朗的音乐环绕着我,

    如同那可爱的白云,

    缓缓升上蓝色的天空!

    逐渐增强,逐渐靠拢,

    要我欣赏,要我倾听,

    要我走进它们中间,

    浸沉在它们的芳香中!

    在它们的波涛里,

    在它们的鸣响中,

    在它们的簇拥下飘动,

    淹没,消融,

    不觉间飞向天庭!\upcite{西洋著名歌剧剧作集2}
    
    这一段描写了其实伊索尔德之死。也可以完全的看出特里斯坦与伊索尔德对这段感情的态度,从一开始,伊索尔德对特里斯坦的爱
    就是坚决的,她不会去估计世俗的眼光,她只想去特里斯坦去的地方,这也与特里斯坦对这段感情的态度截然不同。从这种意义来看
    《爱之死》的这一段与开头描写伊索尔德对特里斯坦“爱的渴望”想呼应。在歌剧最后,与壮观的交响乐同时出现的是伊索尔德平静的
    声音:“在这喧闹的声响里,在这喘息不定的人世间,把一切卷入海底,沉入深渊。”然后,高潮跌落,音乐结束,这意境也是
    伊索尔德在最后那句充满出世意味的歌词中所表达的:“无知无觉是至高无上的快乐。”\upcite{于润洋}

《特里斯坦与伊索尔德》中混杂各种宗教的元素,比如对黑夜的崇敬,对解脱之死的向往。另一方面,瓦格纳又相信爱情是可以战胜一切的。
同时剧中对两个男女主人公心理的刻画,男主人公对爱的态度从抗拒到沉迷到解脱,变化十分复杂,但是女主人公却一如既往的坚决,对真
爱的追求贯穿全剧。而最后的《爱之死》由女主人公的视角来演绎。全剧女主人公一直占据主动,也说明伊索尔德更加接近爱的本真\upcite{王晶}

整部剧围绕这特里斯坦与伊索尔德之爱,表达了很多瓦格纳的思想以及他当时自己的精神状态,而通过查阅的这些资料,以及观看了歌剧的主要部分,
我很难想象瓦格纳是如何做到将一部如此长的戏构思得如此严密,整部歌剧表达的内容以及音乐上的一些东西完全成了一个整体,这也是我了解
了这部歌剧背后的东西最为惊叹的一个地方!
    \bibliographystyle{ieeetr}
    \renewcommand{\refname}{参考文献}
    \bibliography{sample.bib}

\end{document}
