\documentclass[10pt,a4paper]{ctexart}
\usepackage[utf8]{inputenc}
\usepackage{amsmath}
\usepackage{amsfonts}
\usepackage{amssymb}
\usepackage{graphicx}
\usepackage{cite}
\usepackage{natbib}
\newcommand{\upcite}[1]{\textsuperscript{\textsuperscript{\cite{#1}}}}
\author{吴秉哲}
\title{《特里斯坦与伊索尔德》中的爱与悲}
\begin{document}
	\maketitle
	《特里斯坦与伊索尔德》是德国作曲家兼剧作家瓦格纳(1813-1883)成熟时期最有代表性的一部作品,是根据
	爱尔兰的传说撰写的爱情悲剧,瓦格纳创作的基本素材来源于13世纪德国诗人弗列特.冯.斯特拉斯布格的同名叙
	事诗\upcite{百度百科}。虽然其构思来源于古代传奇,但瓦格纳在这部作品中抛弃了以往的魔幻内容,魔鬼,神怪,圣杯骑士在这部作品
	中统统消失不见。瓦格纳将史诗中的情节也大大简化,变成了一部专门描写主人公恋爱内心世界的感情戏,台词大
	多为主人公的内心独白,音乐也大多表现纠缠于爱欲之中的痛苦和幻想。
	
	全局分为三幕,讲述了年轻骑士特里斯坦被派往爱尔兰,去为其叔父—康沃尔国王马克迎娶公主伊索尔德。特里斯坦与伊索尔德
	曾经相识,归涂中公主郁郁寡欢,因为她偷爱着特里斯坦,但他对她却冷淡无情,伊索尔德又想到自己被迫无奈的婚姻愤恨至极,
	宁肯与特里斯坦同归于尽。他命令其女仆在酒里下毒,但是女仆却将毒酒换为了爱情之酒,二人陷入情网。两人在一次深夜幽会中,
	被国王马克撞见,国王手下的骑士梅洛特向特里斯坦挑战,决斗中,特里斯坦身受重伤,他的仆人将其带往其家乡养伤。伊索尔德
	乘船赶到特里斯坦的家乡,特里斯坦非常激动,但还是死于伊索尔德的怀抱中。随后,国王乘船赶到本想成全特里斯坦与伊索尔德
	的爱情,但为时已晚。此时,伊索尔德已唱着爱情之歌,同特里斯坦共相长眠\upcite{麦耶尔}。
	
	整部歌剧是围绕着特里斯坦与伊索尔德两人爱情的产生,发展以及死亡来展开的,自由选择的死亡引领彼此相爱,同时也需要使
 	也需要爱情使两人走向死亡,这就注定了男女主人公命运的悲剧性。
 	
 	下面从几个方面来谈谈《特里斯坦和伊索尔德》创作背景中的悲剧创作元素以及歌剧本身蕴含的爱情与悲情。
 	\section*{创作背景}
 	\subsection*{音乐时代背景}
 	十九世纪中后期的音乐发展正处于浪漫派后期,浪漫主义音乐比起之前的维也纳古典乐派的音乐,更注重感情和形象的表现,相对
 	来说要看轻形式和结构一些。浪漫主义音乐的一个重要特点就是善于表达个人的感情和幻想,尤其强调个人主义的体验。他们热衷
 	于反映内心最强烈情感的爱情主题;倾心于带自传性的,不满现状的忧郁,孤独者的精神\upcite{余志刚}
 	
 	浪漫主义音乐的出现,使音乐走出宫廷,走向民间。新的艺术题材的出现更加丰富了音乐的表现力。这时的艺术家们不像在古典乐派
 	时期沐浴在阳光下,而是沉浸在充满不详预感,倾向于表现内心的思考之中,因而瓦格纳的音乐中充满了矛盾,不安,灵魂的超越
 	以及爱情的伟大等。
 	\subsection*{写作背景}
    瓦格纳在完成《特里斯坦与伊索尔德》时,已经是位在异国漂流近10年的政治流亡者。瓦格纳所处的年代,正是德国社会发生剧烈动荡的
    年代。当时瓦格纳曾热情投入1849年德国资产阶级革命,可是他站在了失败的一方,普鲁士国王的军队镇压了起义,瓦格纳受到通缉和流放
    ,他的那些带有乌托邦色彩的社会理想在社会现实面前被碰得粉碎,这是50年代后期(也就是他创作这部歌剧的时期)经历一场严重精神危机
    的根本原因。这场"革命"不仅没有改变他的命运,却使他更加烦恼,痛苦。他在1855年的《论事本》上写到:“我心里渴望死亡,而生活——是我痛恨的。”
    \upcite{瓦格纳书信集}。这便是50年代后期瓦格纳所面临的精神生活的危机。

    当时,瓦格纳在感情生活上也经历着一场严重的危机。这种危机是瓦格纳同一位有夫之妇—玛蒂尔德\dots韦森东克之间一段没有指望的恋情。瓦格纳
    在逃亡瑞士时,结识了富商欧托\dot韦森东克一家。其夫人玛蒂尔德是个作家,爱好音乐与文学。瓦格纳经常与她讨论文学艺术问题,当时的瓦格纳
    与他的妻子米娜感情不和,几乎闹到决裂。各种因素的作用下,瓦格纳与玛蒂尔德之间产生了恋情。对于瓦格纳自己来说,这是一场不可能有结果的爱情。
    一方面是自己心理过不去,另一方面是社会舆论的压力。当时与妻子不和的瓦格纳独自一人迁居威尼斯,并投入到《特里斯坦与伊索尔德》的创作之中。
    从他的这些经历可以看出这段恋情对这部作品的影响。瓦格纳在给李斯特的信件中曾经说:“虽然我一生从未品尝过爱的甘美,我却要为这个所有梦中最美的梦树立一座纪念碑。在这个梦中,爱得到了永恒的满足。”\upcite{瓦格纳李斯特}
 	\section*{第三幕终场音乐—“爱之死”分析}
    \bibliographystyle{ieeetr}
    \renewcommand{\refname}{参考文献}
    \bibliography{sample.bib}

\end{document}
