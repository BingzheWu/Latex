\documentclass[10pt,a4paper]{ctexart}
\usepackage[utf8]{inputenc}
\usepackage{amsmath}
\usepackage{amsfonts}
\usepackage{amssymb}
\usepackage{graphicx}
\author{Qingyang Kong  孔清扬}
\title{Think Like a Lawyer of American Torts\\ \begin{small}Mid-Term Exercise Submission of American Torts Law\end{small}}
\begin{document}
\maketitle


Though having been studying in Law School of Peking University for two years, this is my first time to learn about torts and American torts. If I say I have no difficult in this course, I must be lying. Every Chinese student may find more or less trouble here, not only because of language barriers, but how to think like an American lawyer of torts. It is much different being a Chinese lawyer following the strict rules and not having much chance to argue on the facts and evidences. 

The most popular and common method in legal education in China is, firstly to give the theories on legal problems and conclusions on legal issues to students, then to express majority opinion on it, and finally to prove that the opinion is correct and reasonable with several cases. As a law school student in China, all you have to do is to follow the professors’ guidance and can easily get the main idea of a legal issue. However, this doesn’t work when learning American torts law. When learning this, only the general idea of the topic is given before been exposed to the cases. All the rules of analysing a legal issue in American torts can only be found by summing up holdings raised by the competitive lawyers on both sides and analysing the reasoning parts of verdicts written by knowledgeable judges. It is difficult for a green hand to hit the point of the problem and figure out clear conclusions by analysing cases in this way. 

What troubles me most is that it is difficult to build a system of analysing. Just following the order of “Duty”, “Negligence”, “Causation”, and “Harm”, doesn’t seem to be helpful when deciding whether one is legally responsible for the result. A more delicate system of thinking shall be built when facing a case. For instance, when talking about “Duty”, the first step is to judge whether a duty at the abstract level exist (duty in law), like, “does a car-driver owe a duty of care to pedestrians?”; then to judge whether the particular claimant is within the scope of the duty of care (duty in fact); finally to judge whether the defendant breached the duty. And there are several facts are generally concerned judging whether the defendant has a certain duty, such as career, time and place of the behavior or negligence happened, special relationship with the plaintiff, and more. Such system is helpful to think clearly, facing a complex case with plaintiff and defendant holding separate ideas against each other. But it is not easy to build one. 

To build such a system of analysing cases, it is not enough to understand the cases in class. Cases in class are just to help finding out where the divergence is and what the major problem is. Only focus on these cases is not likely to get a deep understanding of the legal issues. Academic books and essays are needed and useful, providing opinions from different points of view and summaries of major arguments on the particular issue. But even these books and essays are not enough to make a good and practical decision, sometimes a handout given by government or associations, a custom in certain areas, and other facts shall be all considered as well. Take the duty of care of a nurse as an example, a handbook given by UNISON may help set the standard. This is why the American torts law is difficult for students in law school (like an “ivory tower”) knowing little about the society and the world out of the campus. 

What is more, the ability to figure out closely related facts is necessary. During the discussion in class, I am often confused whether some facts are worth concerned, or having problem with not concerning some facts I personally think is important. For example, in the case of Broadbent v. Broadbent, the fact that Christopher was wearing “floaties” seems to tell that he was not a good swimmer, having no relevant with whether his mother is negligent, but it actually proved that she was not careful enough as a reasonable parent to look after her young child who just started learning to swim and cannot deal with dangerous situations alone. But if the case is about a teenager who is skilled in swimming, it is not necessary to consider whether the kid was wearing “floaties”. In some cases the details are valuable while in others are not. Especially when talking about “Causation”, we Chinese students often make mistakes here, listing remote reasons and details, but not deeply thinking about the facts that really influence the legal judgments which a lawyer of torts may focus on. Pitifully, there is rare standard of what useful details are and what are not. The only way to solve this problem is to think like a lawyer of torts, which means start thinking problems with legal issues, then find the evidences and facts which help to analyse them, rather than thinking of all related facts (no matter how far they are to the result and how closely to the legal issue). For example, when judging whether a result is foreseeable in a certain case, the environment and backgrounds of the cases are important, including but not limited to the facts of weather, time, place and so on. But if one says that defendant’s mother is also legally related because she gave birth to him, this may seem ridiculous, like a Butterfly Effect. This example might be extreme and ridiculous, but is a mirror of the way we think. This process of figuring what are important to the issue may also requires the balance between law and natural justice, which means, there are facts and elements may be worth thinking and can affect the common justice, but are not necessarily to be attentional in the court, and a lawyer of torts shall not ignore them. 

When analysing the cases, it is also easily to be confused that some standards and options of the court changes by time and may seem different in different states. For example, the courts’ attitudes to parent immunity various from time to time and differs in different states. A lawyer in torts may have to figure out how and based on what the views changed, reading plenty of verdicts in different backgrounds, but related closely and focused on the same issue. It is hard to figure out the spirit of the tort law if only view cases with different verdict separately. 

All in all, to think like a lawyer of American torts law ask for the ability of building a system of analysing, figuring the point of the legal issue, finding out the important facts that are closely related to the issue, analysing the separated cases on a same issue and more abilities. For a law school student, the torts law may seem far from the ivory tower we stick to, but actually, every event, no matter big or not, is more or less related to it. And the more one think about the academic but practical problems in torts law, the more happiness he may find and the more reasonable he will act. 


\end{document}